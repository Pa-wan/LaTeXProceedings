% Copyright 2017, Giampiero Salvi <giampi@kth.se>
% electronic version of the proceedings
\documentclass{confproc}
\usepackage{hyperref}
\hypersetup{
%    bookmarks=true,         % show bookmarks bar?
%    unicode=false,          % non-Latin characters in Acrobat’s bookmarks
%    pdftoolbar=true,        % show Acrobat’s toolbar?
%    pdfmenubar=true,        % show Acrobat’s menu?
%    pdffitwindow=false,     % window fit to page when opened
%    pdfstartview={FitH},    % fits the width of the page to the window
    pdftitle={Grounding Language Understanding 2017},    % title
    pdfauthor={Giampiero Salvi, Stéphane Dupont},     % author
%    pdfsubject={Subject},   % subject of the document
%    pdfcreator={Creator},   % creator of the document
%    pdfproducer={Producer}, % producer of the document
    pdfkeywords={speech} {language} {grounding} {affordances}, % list of keywords
%    pdfnewwindow=true,      % links in new window
%    colorlinks=false,       % false: boxed links; true: colored links
%    linkcolor=red,          % color of internal links
%    citecolor=green,        % color of links to bibliography
%    filecolor=magenta,      % color of file links
%    urlcolor=cyan           % color of external links
    plainpages=false         % for the authorindex package
}
\def\theaipage{\string\hyperpage{\thepage}}
\isbn{978-91-7501-080-9}
\howpublished{Electronic version}

\usepackage[utf8]{inputenc}
\usepackage{multicol}

\title{Grounding Language Understanding 2017}
\conferencetype{International Workshop}
\date{Aug 25, 2017}
\address{Stockholm, Sweden}

\editor{Giampiero Salvi, Stéphane Dupont}
\school{KTH Royal Institute of Technology,\\ School of Computer Science and Communication\\ Department of Speech, Music and Hearing\\SE-100 44 Stockholm, Sweden}
\issn{1680-8908}

\makeindex

\begin{document}
\pagenumbering{roman}
\maketitle
%\cleardoublepage
\newpage
\section*{Foreword}\addcontentsline{toc}{section}{Foreword}
Welcome to the International Conference on Auditory-Visual Speech Processing (AVSP) 2011. Sixteen years ago in Bonas, France, the workshop on \emph{Speech reading my man and machine} gathered an interdisciplinary group of researchers involved in various aspects of auditory-visual speech processing. The Bonas workshop may be seen as the start of a long series of meetings in this relatively new research field, since 1997 under the flag of \emph{AVSP}. These meetings have held on to the interdisciplinary spirit of the Bonas workshop, and just as Bonas they have often taken place in more or less isolated countryside locations somewhat tricky to get to (and away from). We believe this is an important factor leading to the creative and familiar atmosphere that has been a trademark of the AVSP conferences.

We hope that this years AVSP, which is the 11\textsuperscript{th} in the series, will be no exception in this respect.  The SIAF (International Executive Campus) Learning Village in Volterra, Tuscany, is idyllically situated in the Tuscan countryside, offering a panoramic view of the sea, and just a short distance from the historical hilltop town of Volterra. The conference is organised by the Department of Speech, Music and Hearing, KTH, Stockholm, Sweden.

These proceedings contain the 28 papers accepted for presentation at AVSP 2011, of which 3 are demo papers. The total number of submissions was 33. The reviewing process required that each paper receive at least two reviews (mostly three reviewers have judged each papers). We are grateful to all of the reviewers listed overleaf for providing their valuable time. 

Additionally, in our programme we have two exceptional invited speakers: Sverre Sjölander --- Professor Emeritus in Zoology, Linköping University, Sweden and Colm Massey --- Head of research and development at AudioMotion, Oxford, UK. In recent years, AVSP keynote speakers have given a good overview about human audiovisual speech processing. In our opinion, Sverre Sjölander's talk can put this subject into a broader perspective, and provide the background for discussions on what is specific for human audiovisual speech processing and what is more general perception. On the other hand, Colm Massey's talk will address the ever growing field of truthful animation of human motion for computer gaming and entertainment. Because this has become a truly important part of society, we believe the AVSP participant will find it interesting to explore how realistic visual speech for computer games is to date, where the bottlenecks are, and what lies ahead in research and development. We believe the keynote speaker's contributions will fit the theme of the conference, where fundamental research in human behaviour and perception meets the more applied research and development in the entertainment industry. 

We are also pleased that AVSP 2011, as with previous years, has attracted a truly interdisciplinary group of delegates. We have papers relating to speech and gesture, speech perception, audiovisual speech synthesis, audiovisual speech recognition and robotics. We encourage participants to take advantage of this heterogeneous group of delegates together with the unique setting of AVSP conferences, where the common meals and the small size of the conference make many fruitful discussions possible. We hope that you have an enjoyable conference and find the time to visit some of the wonderful sites in and around Volterra.

\vspace{1cm}
\noindent The GLU 2017 organising committee

\noindent Stockholm, August 2017.

\newpage
\section*{Scientific Committee}\addcontentsline{toc}{section}{Referees}
\begin{itemize}
\item Leonardo Badino, Italian Institute of Technology, Italy
\item Claude Barras, LIMSI-CNRS, France
\item Tony Belpaeme, Plymouth University, UK
\item Alexandre Bernardino, Instituto Superior Técnico, Lisbon, Portugal
\item Angelo Cangelosi, Plymouth University, UK
\item Javier Civera, Universidad de Zaragoza, Spain
\item Aaron Courville, Universit de Montral, Canada
\item Laurence Devillers, LIMSI-CNRS, France
\item Stéphane Dupont, University of Mons, Belgium
\item Thierry Dutoit, University of Mons, Belgium
\item Begoña García-Zapirain, University of Deusto, Bilbao, Spain
\item Denis Jouvet, Loria, France
\item Robert Legenstein, Graz University of Technology, Austria
\item Mikołaj Leszczuk, AGH Unviersity, Krakow, Poland
\item Manuel Lopes, Instituto Superior Técnico, Lisbon, Portugal
\item Cynthia Matuszek, University of Maryland, Baltimore County, USA
\item Marie-Francine Moens, Katholieke Universiteit Leuven, Heverlee, Belgium
\item Ana C Murillo, Universidad de Zaragoza, Spain
\item Pierre-Yves Oudeyer, Inria, France
\item Michael Spranger, Sony Computer Science Laboratories Inc., Tokyo, Japan
\item Olivier Pietquin, Université de Lille, France
\item Jean Rouat, Universit de Sherbrooke, Canada
\item Marco Sabato Siniscalchi, Kore University, Enna, Italy
\item Giampiero Salvi, KTH Royal Institute of Technology, Stockholm, Sweden
\item José Santos-Victor, Instituto Superior Técnico, Lisbon, Portugal
\item Kamel Smaïli, Université de Lorraine , Nancy, France
\item Hugo Van hamme, University of Leuven, Belgium
\end{itemize}
%\cleardoublepage
\newpage
\tableofcontents

\cleardoublepage
% Keynote speakers
\pagenumbering{arabic}
\refstepcounter{section}\addcontentsline{toc}{section}{Invited papers}
\includepaper{}{Robert Legenstein}{articles/avsp11_submission}
\includepaper{}{Emmanuel Dupoux}{articles/avsp11_submission}
\includepaper{}{Katerina Pastra}{articles/avsp11_submission}

\cleardoublepage
% rest of papers
\refstepcounter{section}\addcontentsline{toc}{section}{Reviewed papers}
\includepaper{Visual Speech Speeds Up Auditory Identification Responses}{Tim Paris, Jeesun Kim, Chris Davis}{articles/avsp11_submission}
\includepaper{Do infants detect A$\rightarrow$V articulator congruency for nonnative click consonants?}{Catherine Best, Christian Kroos, Julia Irwin}{articles/avsp11_submission}
\includepaper{Perceiving Visual Prosody from Point-Light Displays}{Erin Cvejic, Jeesun Kim, Chris Davis}{articles/avsp11_submission}
\includepaper{Binding and unbinding the McGurk effect in audiovisual speech fusion: Follow-up experiments on a new paradigm}{Olha Nahorna, Fr{\'e}d{\'e}ric Berthommier, Jean-Luc Schwartz}{articles/avsp11_submission}
\includepaper{Children’s expression of uncertainty in collaborative and competitive contexts}{Mandy Visser, Emiel Krahmer, Marc Swerts}{articles/avsp11_submission}
\includepaper{The effect of seeing the interlocutor on auditory and visual speech production in noise.}{Michael Fitzpatrick, Jeesun Kim, Chris Davis}{articles/avsp11_submission}
\includepaper{Auditory-Visual Discrimination and Identification of Lexical Tone Within and Across Tone Languages}{Denis Burnham, Virginie Attina, Benjawan Kasisopa}{articles/avsp11_submission}
\includepaper{Audiovisual competition in the perception of counter-expectational questions}{Joan Borr{\`a}s-Comes, Cecilia Pugliesi, Pilar Prieto}{articles/avsp11_submission}
\includepaper{Introducing Visual Target Cost within an Acoustic-Visual Unit-Selection Speech Synthesizer}{Utpala Musti, Vincent Colotte, Asterios Toutios, Slim Ouni}{articles/avsp11_submission}
\includepaper{Auditory and Photo-realistic Audiovisual Speech Synthesis for Dutch}{Wesley Mattheyses, Lukas Latacz, Werner Verhelst}{articles/avsp11_submission}
\includepaper{Realistic Visual Speech Synthesis Based on AAM Features and an Articulatory DBN Model with Constrained Asynchrony}{Peng Wu, Dongmei Jiang, He Zhang, Hichem Sahli}{articles/avsp11_submission}
\includepaper{Talking Heads for Elderly and Alzheimer Patients (THEA): Project Report and Demonstration}{Sascha Fagel}{articles/avsp11_submission}
\includepaper{Improving Naturalness of Visual Speech Synthesis}{Laszlo Czap}{articles/avsp11_submission}
\includepaper{A robotic head using projected animated faces.}{Samer {Al Moubayed}, Simon Alexandersson, Jonas Beskow, Bj{\"o}rn Granstr{\"o}m}{articles/avsp11_submission}
\includepaper{Audiovisual speech processing in visual speech noise}{Kim Jeesun, Chris Davis}{articles/avsp11_submission}
\includepaper{Audiovisual streaming in voicing perception: new evidence for a low level interaction between audio and visual modalities}{Fr{\'e}d{\'e}ric Berthommier, Jean-Luc Schwartz}{articles/avsp11_submission}
\includepaper{An ordinal model of the McGurk illusion}{Tobias Andersen}{articles/avsp11_submission}
\includepaper{Thin slices of head movements during problem solving reveal level of difficulty}{Bart Joosten, Marije {Van Amelsvoort}, Emiel Krahmer, Eric Postma}{articles/avsp11_submission}
\includepaper{Dimensional Mapping of Multimodal Integration on Audiovisual Emotion Perception}{Yoshiko Arimoto, Kazuo Okanoya}{articles/avsp11_submission}
\includepaper{Turn-taking Control Using Gaze in Multiparty Human-Computer Dialogue: Effects of 2D and 3D Displays}{Samer {Al Moubayed}, Gabriel Skantze}{articles/avsp11_submission}
\includepaper{Bilingual Corpus for AVASR using Multiple Sensors and Depth Information}{Georgios Galatas, Gerasimos Potamianos, Dimitrios Kosmopoulos, Chris Mcmurrough, Fillia Makedon}{articles/avsp11_submission}
\includepaper{Kinetic Data for Large-Scale Analysis and Modeling of Face-to-Face Conversation}{Jonas Beskow, Simon Alexandersson, Samer {Al Moubayed}, Jens Edlund, David House}{articles/avsp11_submission}
\includepaper{``Mask-bot'' --- a life-size talking head animation robot for AV speech and human-robot communication research}{Takaaki Kuratate, Brennand Pierce, Gordon Cheng}{articles/avsp11_submission}
\includepaper{Development of Communication Support System using Lip Reading}{Takeshi Saitoh}{articles/avsp11_submission}
\includepaper{Lucia-WebGL:  a Web Based Italian MPEG-4 Talking Head}{Giuseppe Riccardo Leone, Piero Cosi}{articles/avsp11_submission}
%\includepaper{Comparison of Viseme Definitions for Visual Speech Recognition}{Luca Cappelletta, Naomi Harte}{articles/avsp11_submission}
\includepaper[0 -15]{Improved Detection of Ball Hit Events in a Tennis Game Using Multimodal Information}{Qiang Huang, Stephen Cox}{articles/avsp11_submission}
\includepaper{Speech-driven lip motion generation for tele-operated humanoid robots}{Carlos Ishi, Chaoran Liu, Hiroshi Ishiguro, Norihiro Hagita}{articles/avsp11_submission}
\includepaper{On the Audiovisual Asynchrony of Speech}{Laszlo Czap}{articles/avsp11_submission}


\section*{Author Index}\addcontentsline{toc}{section}{Author Index}
\begin{multicols}{2}
\printauthorindex
\end{multicols}
\end{document}

% do not remove (emacs configuration)
% Local variables:
% enable-local-variables: t
% ispell-local-dictionary: "british"
% mode: latex
% eval: (flyspell-mode)
% eval: (flyspell-buffer)
% End:
