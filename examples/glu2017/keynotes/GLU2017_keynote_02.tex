\documentclass{article}
\usepackage[utf8]{inputenc}
\usepackage{INTERSPEECH_v2}
\title{Keynote 2:\\Recursion all the way: in Language, Action and Semantic Association}
\name{Katerina Pastra}
\address{Cognitive Systems Research Institute\\Athens, Greece}
\email{kpastra@csri.gr}

\begin{document}
\maketitle
\thispagestyle{empty}
\begin{abstract}
The phenomenon of recursion has been considered to be a unique characteristic of human language. However, increasing evidence in Neuroscience points to the fact that a fundamental syntactic mechanism is shared between language and action, both of which have a hieararchical and compositional organisation; Broca's area has been suggested as the neural locus of this mechanism.  In this talk, we will present the first formal specification of action with biological bases, the Minimalist Grammar of Action. The grammar allows the development of  generative computational models for action in the motor and visuomotor space. Through the grammar, we present examples of recursion in the action space and how a generative action perception/execution system may account for the phenomenon. Furthemore, we go a step further, arguing that recursion is not a phenomenon that arises in the language or action space only; it is a phenomenon that lies in any 'syntactic' activity, in any space that comprises 'merging' of elements into more and more complex units. We present recursion in the Semantic Association space (Semantic Memory), through the PRAXICON, the first ever recursive and referential semantic network. We will demonstrate the importance of the network in generalisation and reasoning for a number of applications and discuss the interdisciplinary implications of our argument on recursion. 
\end{abstract}
\vfill{}
\end{document}
