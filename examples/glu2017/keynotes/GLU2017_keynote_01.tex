\documentclass{article}
\usepackage[utf8]{inputenc}
\usepackage{INTERSPEECH_v2}
\title{Keynote 1: Binding through assemblies in the human brain: Recent data and models}
\name{Robert Legenstein}
\address{Institute for Theoretical Computer Science\\Graz University of Technology\\Austria}
\email{robert.legenstein@igi.tugraz.at}

\begin{document}
\maketitle
\thispagestyle{empty}
\begin{abstract}
Recent experimental data has provided valuable insights into the representation of concepts and language in the human brain. Electrode recordings from the human brain suggest that concepts are represented in the medial temporal lobe (MTL) through sparse sets of neurons (assemblies). Further, fMRI recordings from the human brain suggest that specific subregions of the temporal cortex are dedicated to the representation of specific roles (e.g., subject or object) of concepts in a sentence or visually presented episode. We propose that quickly recruited assemblies of neurons in these subregions act as pointers  to previously created assemblies that represent concepts. We refer to these pointers as assembly pointers. In this computational architecture, the thematic role of a word in a sentence or episode (e.g., agent or patient) is bound to a concrete filler (e.g., a word or concept) during language processing, an operation that has been termed variable binding. We provide a proof of principle that the resulting model for binding through assembly pointers can be implemented in networks of spiking neurons, and supports basic operations of brain computations, such as structured recall and the flexible handling of information.
\end{abstract}
\vfill{}
\end{document}
