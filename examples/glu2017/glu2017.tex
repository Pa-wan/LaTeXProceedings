% Copyright 2017, Giampiero Salvi <giampi@kth.se>
% electronic version of the proceedings
\documentclass{confproc}
\usepackage{hyperref}
\hypersetup{
%    bookmarks=true,         % show bookmarks bar?
%    unicode=false,          % non-Latin characters in Acrobat’s bookmarks
%    pdftoolbar=true,        % show Acrobat’s toolbar?
%    pdfmenubar=true,        % show Acrobat’s menu?
%    pdffitwindow=false,     % window fit to page when opened
%    pdfstartview={FitH},    % fits the width of the page to the window
    pdftitle={Grounding Language Understanding 2017},    % title
    pdfauthor={Giampiero Salvi, Stéphane Dupont},     % author
%    pdfsubject={Subject},   % subject of the document
%    pdfcreator={Creator},   % creator of the document
%    pdfproducer={Producer}, % producer of the document
    pdfkeywords={speech} {language} {grounding} {affordances}, % list of keywords
%    pdfnewwindow=true,      % links in new window
%    colorlinks=false,       % false: boxed links; true: colored links
%    linkcolor=red,          % color of internal links
%    citecolor=green,        % color of links to bibliography
%    filecolor=magenta,      % color of file links
%    urlcolor=cyan           % color of external links
    plainpages=false         % for the authorindex package
}
\def\theaipage{\string\hyperpage{\thepage}}
\isbn{XXX-XX-XXXX-XXX-X}
\howpublished{Electronic version}

\usepackage[utf8]{inputenc}
\usepackage{multicol}

\title{Grounding Language Understanding 2017}
\conferencetype{International Workshop}
\date{Aug 25, 2017}
\address{Stockholm, Sweden}

\editor{Giampiero Salvi and Stéphane Dupont}
\school{KTH Royal Institute of Technology, Stockholm, Sweden\\University of Mons, Belgium}
\issn{XXXX-XXXX}

\makeindex

\begin{document}
\pagenumbering{roman}
\maketitle
%\cleardoublepage
\newpage
\section*{Foreword}\addcontentsline{toc}{section}{Foreword}
Welcome to the First International Conference on Grounding Language Understanding (GLU 2017).

\section*{Acknowledgements}
We would like to thank Prof. Jean Rouat from Université de Sherbrooke, Canada, for his advice and help in organising this workshop. Without his contribution, the workshop would have assumed a very different form. We would also like to thank the members of the scientific committee for their help with ensuring the high standards of publication in these workshop proceedings.

\vspace{1cm}
\noindent The GLU 2017 organising committee

\noindent Stockholm, August 2017.

\newpage
\section*{Scientific Committee}\addcontentsline{toc}{section}{Referees}
\begin{itemize}
\item Leonardo Badino, Italian Institute of Technology, Italy
\item Claude Barras, LIMSI-CNRS, France
\item Tony Belpaeme, Plymouth University, UK
\item Alexandre Bernardino, Instituto Superior Técnico, Lisbon, Portugal
\item Angelo Cangelosi, Plymouth University, UK
\item Javier Civera, Universidad de Zaragoza, Spain
\item Aaron Courville, Universit de Montral, Canada
\item Laurence Devillers, LIMSI-CNRS, France
\item Stéphane Dupont, University of Mons, Belgium
\item Thierry Dutoit, University of Mons, Belgium
\item Begoña García-Zapirain, University of Deusto, Bilbao, Spain
\item Denis Jouvet, Loria, France
\item Robert Legenstein, Graz University of Technology, Austria
\item Mikołaj Leszczuk, AGH Unviersity, Krakow, Poland
\item Manuel Lopes, Instituto Superior Técnico, Lisbon, Portugal
\item Cynthia Matuszek, University of Maryland, Baltimore County, USA
\item Marie-Francine Moens, Katholieke Universiteit Leuven, Heverlee, Belgium
\item Ana C Murillo, Universidad de Zaragoza, Spain
\item Pierre-Yves Oudeyer, Inria, France
\item Michael Spranger, Sony Computer Science Laboratories Inc., Tokyo, Japan
\item Olivier Pietquin, Université de Lille, France
\item Jean Rouat, Universit de Sherbrooke, Canada
\item Marco Sabato Siniscalchi, Kore University, Enna, Italy
\item Giampiero Salvi, KTH Royal Institute of Technology, Stockholm, Sweden
\item José Santos-Victor, Instituto Superior Técnico, Lisbon, Portugal
\item Kamel Smaïli, Université de Lorraine , Nancy, France
\item Hugo Van hamme, University of Leuven, Belgium
\end{itemize}
%\cleardoublepage
\newpage
\tableofcontents

\cleardoublepage
% Keynote speakers
\pagenumbering{arabic}
\refstepcounter{section}\addcontentsline{toc}{section}{Invited papers}
\includepaper{Binding through assemblies in the human brain: Recent data and models}{Robert Legenstein}{keynotes/GLU2017_keynote_01} % fix info
%Recent experimental data has provided valuable insights into the representation of concepts and language in the human brain. Electrode recordings from the human brain suggest that concepts are represented in the medial temporal lobe (MTL) through sparse sets of neurons (assemblies). Further, fMRI recordings from the human brain suggest that specific subregions of the temporal cortex are dedicated to the representation of specific roles (e.g., subject or object) of concepts in a sentence or visually presented episode. We propose that quickly recruited assemblies of neurons in these subregions act as pointers  to previously created assemblies that represent concepts. We refer to these pointers as assembly pointers. In this computational architecture, the thematic role of a word in a sentence or episode (e.g., agent or patient) is bound to a concrete filler (e.g., a word or concept) during language processing, an operation that has been termed variable binding. We provide a proof of principle that the resulting model for binding through assembly pointers can be implemented in networks of spiking neurons, and supports basic operations of brain computations, such as structured recall and the flexible handling of information.
\includepaper{Recursion all the way: in Language, Action and Semantic Association}{Katerina Pastra}{keynotes/GLU2017_keynote_02} % fix info
% The phenomenon of recursion has been considered to be a unique characteristic of human language. However, increasing evidence in Neuroscience points to the fact that a fundamental syntactic mechanism is shared between language and action, both of which have a hieararchical and compositional organisation; Broca's area has been suggested as the neural locus of this mechanism.  In this talk, we will present the first formal specification of action with biological bases, the Minimalist Grammar of Action. The grammar allows the development of  generative computational models for action in the motor and visuomotor space. Through the grammar, we present examples of recursion in the action space and how a generative action perception/execution system may account for the phenomenon. Furthemore, we go a step further, arguing that recursion is not a phenomenon that arises in the language or action space only; it is a phenomenon that lies in any 'syntactic' activity, in any space that comprises 'merging' of elements into more and more complex units. We present recursion in the Semantic Association space (Semantic Memory), through the PRAXICON, the first ever recursive and referential semantic network. We will demonstrate the importance of the network in generalisation and reasoning for a number of applications and discuss the interdisciplinary implications of our argument on recursion. 
\includepaper{}{Emmanuel Dupoux}{keynotes/GLU2017_paper_10} % fix info
\cleardoublepage
% rest of papers
\refstepcounter{section}\addcontentsline{toc}{section}{Reviewed papers}
\includepaper{Communication with Speech and Gestures: Applications of Recurrent Neural Networks to Robot Language Learning}{Alexandre Antunes, Gabriella Pizzuto and Angelo Cangelosi}{papers/GLU2017_paper_1}
\includepaper{Towards a Knowledge Graph based Speech Interface}{Ashwini Jaya Kumar, Sören Auer, Christoph Schmidt, Joachim Köhler}{papers/GLU2017_paper_2}
\includepaper{Relational Symbol Grounding through Affordance Learning: An Overview of the ReGround Project}{Laura Antanas, Jesse Davis, Luc {De Raedt}, Amy Loutfi, Andreas Persson, Alessandro Saffiotti, Deniz Yuret, Ozan Arkan Can, Emre Unal, Pedro {Zuidberg Dos Martires}}{papers/GLU2017_paper_3}
\includepaper{Building a Multimodal Lexicon: Lessons from Infants' Learning of Body Part Words}{Rana Abu-Zhaya, Amanda Seidl, Ruth Tincoff and Alejandrina Cristia}{papers/GLU2017_paper_4}
\includepaper{Sparse Autoencoder Based Semi-Supervised Learning for Phone Classification with Limited Annotations}{Akash {Kumar Dhaka}, Giampiero Salvi}{papers/GLU2017_paper_5}
\includepaper{Partitioning of Posteriorgrams Using Siamese Models for Unsupervised Acoustic Modelling}{Arvid {Fahlström Myrman}, Giampiero Salvi}{papers/GLU2017_paper_6}
\includepaper{Comparison of Effect of Speaker's Eye Gaze on Selection of Next Speaker between Native- and Second-Language Conversations}{Koki Ijuin, Takato Yamashita, Tsuneo Kato, Seiichi Yamamoto}{papers/GLU2017_paper_7}
\includepaper{Language is Not About Language: Towards Formalizing the Role of Extra-Linguistic Factors in Human and Machine Language Acquisition and Communication}{Okko Räsänen}{papers/GLU2017_paper_9}
\includepaper{SPEECH-COCO: 600k Visually Grounded Spoken Captions Aligned to MSCOCO Data Set}{William Havard, Laurent Besacier, Olivier Rosec}{papers/GLU2017_paper_10}
\includepaper{Vision-based Active Speaker Detection in Multiparty Interaction}{Kalin Stefanov, Jonas Beskow, Giampiero Salvi}{papers/GLU2017_paper_11}
\includepaper{Automatic Speaker's Role Classification With a Bottom-up Acoustic Feature Selection}{Vered Silber-Varod, Anat Lerner, Oliver Jokisch}{papers/GLU2017_paper_13}
\includepaper{Analysis of Audio-Visual Features for Unsupervised Speech Recognition}{Jennifer Drexler, James Glass}{papers/GLU2017_paper_14}
\includepaper{Visually Grounded Word Embeddings and Richer Visual Features for Improving Multimodal Neural Machine Translation}{Jean-Benoit Delbrouck, Stéphane Dupont, Omar Seddati}{papers/GLU2017_paper_15}
\includepaper{Proposal of a Generative Model of Event-based Representations for Grounded Language Understanding}{Simon Brodeur, Luca Celotti, Jean Rouat}{papers/GLU2017_paper_16}
\includepaper{Finding Regions of Interest from Multimodal Human-Robot Interactions}{Pablo Azagra, Javier Civera, Ana C. Murillo}{papers/GLU2017_paper_17}
\includepaper{Enhancing Reference Resolution in Dialogue Using Participant Feedback}{Todd Shore, Gabriel Skantze}{papers/GLU2017_paper_18}
\includepaper{Interactive Robot Learning of Gestures, Language and Affordances}{Giovanni Saponaro, Lorenzo Jamone, Alexandre Bernardino, Giampiero Salvi}{papers/GLU2017_paper_20}
\includepaper{Grounding Imperatives to Actions is Not Enough: A Challenge for Grounded NLU for Robots from Human-Human data}{Julian Hough, Sina Zarriess, David Schlangen}{papers/GLU2017_paper_21}


\section*{Author Index}\addcontentsline{toc}{section}{Author Index}
\begin{multicols}{2}
\printauthorindex
\end{multicols}
\end{document}

% do not remove (emacs configuration)
% Local variables:
% enable-local-variables: t
% ispell-local-dictionary: "british"
% mode: latex
% eval: (flyspell-mode)
% eval: (flyspell-buffer)
% End:
