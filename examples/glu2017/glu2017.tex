% Copyright 2017, Giampiero Salvi <giampi@kth.se>
% electronic version of the proceedings
\documentclass{confproc}
\usepackage[utf8]{inputenc}
\usepackage{hyperref}
\hypersetup{
%    bookmarks=true,         % show bookmarks bar?
%    unicode=false,          % non-Latin characters in Acrobat’s bookmarks
%    pdftoolbar=true,        % show Acrobat’s toolbar?
%    pdfmenubar=true,        % show Acrobat’s menu?
%    pdffitwindow=false,     % window fit to page when opened
%    pdfstartview={FitH},    % fits the width of the page to the window
    pdftitle={Grounding Language Understanding 2017},    % title
    pdfauthor={Giampiero Salvi, Stéphane Dupont},     % author
%    pdfsubject={Subject},   % subject of the document
%    pdfcreator={Creator},   % creator of the document
%    pdfproducer={Producer}, % producer of the document
    pdfkeywords={speech} {language} {grounding} {affordances}, % list of keywords
%    pdfnewwindow=true,      % links in new window
%    colorlinks=false,       % false: boxed links; true: colored links
%    linkcolor=red,          % color of internal links
%    citecolor=green,        % color of links to bibliography
%    filecolor=magenta,      % color of file links
%    urlcolor=cyan           % color of external links
    plainpages=false         % for the authorindex package
}
\def\theaipage{\string\hyperpage{\thepage}}
%\isbn{XXX-XX-XXXX-XXX-X}
\howpublished{Electronic version}

\usepackage{multicol}

\title{Grounding Language Understanding (GLU~2017)}
\conferencetype{First International Workshop}
\date{Aug 25, 2017}
\address{Stockholm, Sweden}

\editor{Giampiero Salvi$^1$ and Stéphane Dupont$^2$}
\school{$^{1)}$ KTH Royal Institute of Technology, Stockholm, Sweden\\$^{2)}$ University of Mons, Belgium}
\publishedby{KTH Royal Institute of Technology, Stockholm, Sweden}
\issn{1613-0073}

\makeindex

\begin{document}

\pagenumbering{roman}
\maketitle
%\cleardoublepage
\newpage
\section*{Foreword}\addcontentsline{toc}{section}{Foreword}
Welcome to the First International Conference on Grounding Language Understanding (GLU~2017).

In a world where robots and intelligent systems are sharing the same environment with humans, new challenges in human-machine interaction emerge.
Problems of online grounded language acquisition, understanding and visualisation in the context of mixed-initiative human-agent interactions become essential if humans and machines are to understand each other and co-exist smoothly.
The scientific community is aware of the challenge and a number of venues have been created to discuss progress.
Noteworthy, for example, are the Google workshop on Language Grounding for Robotics happening three weeks before GLU~2017 in Vancouver, Canada, or the 2nd Workshop on Language Learning at 2017 IEEE ICDL-EPIROB happening in Lisbon, Portugal in September 2017, or, finally, the Third International Workshop on Intrinsically Motivated Open-ended Learning, happening in Rome, Italy in October 2017.

Funding agencies are also focusing on this challenge.
An example is the CHIST-ERA call for Human Language Understanding: Grounding Language Learning that funded a number of projects including IGLU (Interactive Grounded Language Understanding).
The GLU~2017 workshop originates from the need to share results between the projects that were funded within this call, but quickly assumed a much wider scope.
Researchers from 17 countries spanning 4 continents are attending the workshop.
More importantly, many different domains are included, ranging from developmental robotics, speech technology, machine learning, cognitive and systems neuroscience.

We received 21 submissions to the workshop. All submissions received at least three reviews from the Scientific Committee (in some cases four reviews were performed). Following the review suggestions and scores, 12 papers were accepted for oral presentation, 6 for poster presentation and 3 were rejected.

We were also fortunate to welcome three outstanding keynote speakers to the workshop:
\begin{description}
\item[Robert Legenstein] is Associate Professor at the Institute for Theoretical Computer Science at the Graz University of Technology, Austria and Associate Editor for IEEE Transactions on Neural Networks. His research interests range from computation and learning in networks of spiking neurons to computational complexity theory.
\item[Katerina Pastra] is the director of the Cognitive Systems Research Institute, Athens, Greece. She was coordinator for a number of European projects related to the topic, such as Poeticon and Poeticon++. She brought the linguistic perspective to the problem of grounding.
\item[Emmanuel Dupoux] is director of studies at École des Hautes Etudes en Sciences Sociales, Laboratoire de Science Cognitive et Psycholinguistique, Paris, France. His focus on early acquisition of linguistic and social skills in infants brought yet another angle to the workshop discussions.
\end{description}
\section*{Acknowledgements}
We would like to thank Prof. \textbf{Jean Rouat} from Université de Sherbrooke, Canada, for his advice and help in organising this workshop. Without his contribution, the workshop would have assumed a very different form. We would also like to thank the members of the scientific committee for their help with ensuring the high standards of publication in these workshop proceedings.

\vspace{1cm}
\noindent The GLU 2017 organising committee

\noindent Stockholm, August 2017.

\newpage
\section*{Scientific Committee}\addcontentsline{toc}{section}{Referees}
\begin{itemize}
\item Leonardo Badino, Italian Institute of Technology, Italy
\item Claude Barras, LIMSI-CNRS, France
\item Tony Belpaeme, Plymouth University, UK
\item Alexandre Bernardino, Instituto Superior Técnico, Lisbon, Portugal
\item Angelo Cangelosi, Plymouth University, UK
\item Javier Civera, Universidad de Zaragoza, Spain
\item Aaron Courville, Universit de Montral, Canada
\item Laurence Devillers, LIMSI-CNRS, France
\item Stéphane Dupont, University of Mons, Belgium
\item Thierry Dutoit, University of Mons, Belgium
\item Begoña García-Zapirain, University of Deusto, Bilbao, Spain
\item Denis Jouvet, Loria, France
\item Robert Legenstein, Graz University of Technology, Austria
\item Mikołaj Leszczuk, AGH Unviersity, Krakow, Poland
\item Manuel Lopes, Instituto Superior Técnico, Lisbon, Portugal
\item Cynthia Matuszek, University of Maryland, Baltimore County, USA
\item Marie-Francine Moens, Katholieke Universiteit Leuven, Heverlee, Belgium
\item Ana C Murillo, Universidad de Zaragoza, Spain
\item Pierre-Yves Oudeyer, Inria, France
\item Michael Spranger, Sony Computer Science Laboratories Inc., Tokyo, Japan
\item Olivier Pietquin, Université de Lille, France
\item Jean Rouat, Universit de Sherbrooke, Canada
\item Marco Sabato Siniscalchi, Kore University, Enna, Italy
\item Giampiero Salvi, KTH Royal Institute of Technology, Stockholm, Sweden
\item José Santos-Victor, Instituto Superior Técnico, Lisbon, Portugal
\item Kamel Smaïli, Université de Lorraine , Nancy, France
\item Kalin Stefanov, KTH Royal Institute of Technology, Stockholm, Sweden
\item Hugo Van hamme, University of Leuven, Belgium
\end{itemize}
%\cleardoublepage
\newpage
\tableofcontents

\cleardoublepage
% Keynote speakers
\pagenumbering{arabic}
\begin{proceedingssession}{Invited papers}
\includepaper{Binding through assemblies in the human brain: Recent data and models}{Robert Legenstein}{keynotes/GLU2017_keynote_01}
\includepaper{Recursion all the way: in Language, Action and Semantic Association}{Katerina Pastra}{keynotes/GLU2017_keynote_02}
\includepaper{Towards Autonomous Language Learning}{Emmanuel Dupoux}{keynotes/GLU2017_keynote_03}
\end{proceedingssession}
\cleardoublepage
% rest of papers
\begin{proceedingssession}{Regular papers}
\includepaper{Communication with Speech and Gestures: Applications of Recurrent Neural Networks to Robot Language Learning}{Alexandre Antunes, Gabriella Pizzuto and Angelo Cangelosi}{papers/GLU2017_paper_1}
\includepaper{Towards a Knowledge Graph based Speech Interface}{Ashwini Jaya Kumar, Sören Auer, Christoph Schmidt, Joachim Köhler}{papers/GLU2017_paper_2}
\includepaper{Relational Symbol Grounding through Affordance Learning: An Overview of the ReGround Project}{Laura Antanas, Jesse Davis, Luc {De Raedt}, Amy Loutfi, Andreas Persson, Alessandro Saffiotti, Deniz Yuret, Ozan Arkan Can, Emre Unal, Pedro {Zuidberg Dos Martires}}{papers/GLU2017_paper_3}
\includepaper{Building a Multimodal Lexicon: Lessons from Infants' Learning of Body Part Words}{Rana Abu-Zhaya, Amanda Seidl, Ruth Tincoff and Alejandrina Cristia}{papers/GLU2017_paper_4}
\includepaper{Sparse Autoencoder Based Semi-Supervised Learning for Phone Classification with Limited Annotations}{Akash {Kumar Dhaka}, Giampiero Salvi}{papers/GLU2017_paper_5}
\includepaper{Partitioning of Posteriorgrams Using Siamese Models for Unsupervised Acoustic Modelling}{Arvid {Fahlström Myrman}, Giampiero Salvi}{papers/GLU2017_paper_6}
\includepaper{Comparison of Effect of Speaker's Eye Gaze on Selection of Next Speaker between Native- and Second-Language Conversations}{Koki Ijuin, Takato Yamashita, Tsuneo Kato, Seiichi Yamamoto}{papers/GLU2017_paper_7}
\includepaper{Language is Not About Language: Towards Formalizing the Role of Extra-Linguistic Factors in Human and Machine Language Acquisition and Communication}{Okko Räsänen}{papers/GLU2017_paper_9}
\includepaper{SPEECH-COCO: 600k Visually Grounded Spoken Captions Aligned to MSCOCO Data Set}{William Havard, Laurent Besacier, Olivier Rosec}{papers/GLU2017_paper_10}
\includepaper{Vision-based Active Speaker Detection in Multiparty Interaction}{Kalin Stefanov, Jonas Beskow, Giampiero Salvi}{papers/GLU2017_paper_11}
\includepaper{Automatic Speaker's Role Classification With a Bottom-up Acoustic Feature Selection}{Vered Silber-Varod, Anat Lerner, Oliver Jokisch}{papers/GLU2017_paper_13}
\includepaper{Analysis of Audio-Visual Features for Unsupervised Speech Recognition}{Jennifer Drexler, James Glass}{papers/GLU2017_paper_14}
\includepaper{Visually Grounded Word Embeddings and Richer Visual Features for Improving Multimodal Neural Machine Translation}{Jean-Benoit Delbrouck, Stéphane Dupont, Omar Seddati}{papers/GLU2017_paper_15}
\includepaper{Proposal of a Generative Model of Event-based Representations for Grounded Language Understanding}{Simon Brodeur, Luca Celotti, Jean Rouat}{papers/GLU2017_paper_16}
\includepaper{Finding Regions of Interest from Multimodal Human-Robot Interactions}{Pablo Azagra, Javier Civera, Ana C. Murillo}{papers/GLU2017_paper_17}
\includepaper{Enhancing Reference Resolution in Dialogue Using Participant Feedback}{Todd Shore, Gabriel Skantze}{papers/GLU2017_paper_18}
\includepaper{Interactive Robot Learning of Gestures, Language and Affordances}{Giovanni Saponaro, Lorenzo Jamone, Alexandre Bernardino, Giampiero Salvi}{papers/GLU2017_paper_20}
\includepaper{Grounding Imperatives to Actions is Not Enough: A Challenge for Grounded NLU for Robots from Human-Human data}{Julian Hough, Sina Zarriess, David Schlangen}{papers/GLU2017_paper_21}

\end{proceedingssession}

\section*{Author Index}\addcontentsline{toc}{section}{Author Index}
\begin{multicols}{2}
\printauthorindex
\end{multicols}
\end{document}

% do not remove (emacs configuration)
% Local variables:
% enable-local-variables: t
% ispell-local-dictionary: "british"
% mode: latex
% eval: (flyspell-mode)
% eval: (flyspell-buffer)
% End:
